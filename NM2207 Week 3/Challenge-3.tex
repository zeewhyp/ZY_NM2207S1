% Options for packages loaded elsewhere
\PassOptionsToPackage{unicode}{hyperref}
\PassOptionsToPackage{hyphens}{url}
%
\documentclass[
]{article}
\usepackage{amsmath,amssymb}
\usepackage{iftex}
\ifPDFTeX
  \usepackage[T1]{fontenc}
  \usepackage[utf8]{inputenc}
  \usepackage{textcomp} % provide euro and other symbols
\else % if luatex or xetex
  \usepackage{unicode-math} % this also loads fontspec
  \defaultfontfeatures{Scale=MatchLowercase}
  \defaultfontfeatures[\rmfamily]{Ligatures=TeX,Scale=1}
\fi
\usepackage{lmodern}
\ifPDFTeX\else
  % xetex/luatex font selection
\fi
% Use upquote if available, for straight quotes in verbatim environments
\IfFileExists{upquote.sty}{\usepackage{upquote}}{}
\IfFileExists{microtype.sty}{% use microtype if available
  \usepackage[]{microtype}
  \UseMicrotypeSet[protrusion]{basicmath} % disable protrusion for tt fonts
}{}
\makeatletter
\@ifundefined{KOMAClassName}{% if non-KOMA class
  \IfFileExists{parskip.sty}{%
    \usepackage{parskip}
  }{% else
    \setlength{\parindent}{0pt}
    \setlength{\parskip}{6pt plus 2pt minus 1pt}}
}{% if KOMA class
  \KOMAoptions{parskip=half}}
\makeatother
\usepackage{xcolor}
\usepackage[margin=1in]{geometry}
\usepackage{color}
\usepackage{fancyvrb}
\newcommand{\VerbBar}{|}
\newcommand{\VERB}{\Verb[commandchars=\\\{\}]}
\DefineVerbatimEnvironment{Highlighting}{Verbatim}{commandchars=\\\{\}}
% Add ',fontsize=\small' for more characters per line
\usepackage{framed}
\definecolor{shadecolor}{RGB}{248,248,248}
\newenvironment{Shaded}{\begin{snugshade}}{\end{snugshade}}
\newcommand{\AlertTok}[1]{\textcolor[rgb]{0.94,0.16,0.16}{#1}}
\newcommand{\AnnotationTok}[1]{\textcolor[rgb]{0.56,0.35,0.01}{\textbf{\textit{#1}}}}
\newcommand{\AttributeTok}[1]{\textcolor[rgb]{0.13,0.29,0.53}{#1}}
\newcommand{\BaseNTok}[1]{\textcolor[rgb]{0.00,0.00,0.81}{#1}}
\newcommand{\BuiltInTok}[1]{#1}
\newcommand{\CharTok}[1]{\textcolor[rgb]{0.31,0.60,0.02}{#1}}
\newcommand{\CommentTok}[1]{\textcolor[rgb]{0.56,0.35,0.01}{\textit{#1}}}
\newcommand{\CommentVarTok}[1]{\textcolor[rgb]{0.56,0.35,0.01}{\textbf{\textit{#1}}}}
\newcommand{\ConstantTok}[1]{\textcolor[rgb]{0.56,0.35,0.01}{#1}}
\newcommand{\ControlFlowTok}[1]{\textcolor[rgb]{0.13,0.29,0.53}{\textbf{#1}}}
\newcommand{\DataTypeTok}[1]{\textcolor[rgb]{0.13,0.29,0.53}{#1}}
\newcommand{\DecValTok}[1]{\textcolor[rgb]{0.00,0.00,0.81}{#1}}
\newcommand{\DocumentationTok}[1]{\textcolor[rgb]{0.56,0.35,0.01}{\textbf{\textit{#1}}}}
\newcommand{\ErrorTok}[1]{\textcolor[rgb]{0.64,0.00,0.00}{\textbf{#1}}}
\newcommand{\ExtensionTok}[1]{#1}
\newcommand{\FloatTok}[1]{\textcolor[rgb]{0.00,0.00,0.81}{#1}}
\newcommand{\FunctionTok}[1]{\textcolor[rgb]{0.13,0.29,0.53}{\textbf{#1}}}
\newcommand{\ImportTok}[1]{#1}
\newcommand{\InformationTok}[1]{\textcolor[rgb]{0.56,0.35,0.01}{\textbf{\textit{#1}}}}
\newcommand{\KeywordTok}[1]{\textcolor[rgb]{0.13,0.29,0.53}{\textbf{#1}}}
\newcommand{\NormalTok}[1]{#1}
\newcommand{\OperatorTok}[1]{\textcolor[rgb]{0.81,0.36,0.00}{\textbf{#1}}}
\newcommand{\OtherTok}[1]{\textcolor[rgb]{0.56,0.35,0.01}{#1}}
\newcommand{\PreprocessorTok}[1]{\textcolor[rgb]{0.56,0.35,0.01}{\textit{#1}}}
\newcommand{\RegionMarkerTok}[1]{#1}
\newcommand{\SpecialCharTok}[1]{\textcolor[rgb]{0.81,0.36,0.00}{\textbf{#1}}}
\newcommand{\SpecialStringTok}[1]{\textcolor[rgb]{0.31,0.60,0.02}{#1}}
\newcommand{\StringTok}[1]{\textcolor[rgb]{0.31,0.60,0.02}{#1}}
\newcommand{\VariableTok}[1]{\textcolor[rgb]{0.00,0.00,0.00}{#1}}
\newcommand{\VerbatimStringTok}[1]{\textcolor[rgb]{0.31,0.60,0.02}{#1}}
\newcommand{\WarningTok}[1]{\textcolor[rgb]{0.56,0.35,0.01}{\textbf{\textit{#1}}}}
\usepackage{graphicx}
\makeatletter
\def\maxwidth{\ifdim\Gin@nat@width>\linewidth\linewidth\else\Gin@nat@width\fi}
\def\maxheight{\ifdim\Gin@nat@height>\textheight\textheight\else\Gin@nat@height\fi}
\makeatother
% Scale images if necessary, so that they will not overflow the page
% margins by default, and it is still possible to overwrite the defaults
% using explicit options in \includegraphics[width, height, ...]{}
\setkeys{Gin}{width=\maxwidth,height=\maxheight,keepaspectratio}
% Set default figure placement to htbp
\makeatletter
\def\fps@figure{htbp}
\makeatother
\setlength{\emergencystretch}{3em} % prevent overfull lines
\providecommand{\tightlist}{%
  \setlength{\itemsep}{0pt}\setlength{\parskip}{0pt}}
\setcounter{secnumdepth}{-\maxdimen} % remove section numbering
\ifLuaTeX
  \usepackage{selnolig}  % disable illegal ligatures
\fi
\IfFileExists{bookmark.sty}{\usepackage{bookmark}}{\usepackage{hyperref}}
\IfFileExists{xurl.sty}{\usepackage{xurl}}{} % add URL line breaks if available
\urlstyle{same}
\hypersetup{
  pdftitle={Challenge-3},
  pdfauthor={Phua Zong Yao},
  hidelinks,
  pdfcreator={LaTeX via pandoc}}

\title{Challenge-3}
\author{Phua Zong Yao}
\date{2023-08-28}

\begin{document}
\maketitle

\hypertarget{i.-questions}{%
\subsection{I. Questions}\label{i.-questions}}

\hypertarget{question-1-emoji-expressions}{%
\paragraph{Question 1: Emoji
Expressions}\label{question-1-emoji-expressions}}

Imagine you're analyzing social media posts for sentiment analysis. If
you were to create a variable named ``postSentiment'' to store the
sentiment of a post using emojis (😄 for positive, 😐 for neutral, 😢
for negative), what data type would you assign to this variable? Why?
(\emph{narrative type question, no code required})

\textbf{Solution:} I would assign the data type `character' to the
variable `postSentiment'. In this case, emojis are considered to be
textual characters. Using the `character' data type allows me to easily
store and move the emojis as if they were groups of text.

\hypertarget{question-2-hashtag-havoc}{%
\paragraph{Question 2: Hashtag Havoc}\label{question-2-hashtag-havoc}}

In a study on trending hashtags, you want to store the list of hashtags
associated with a post. What data type would you choose for the variable
``postHashtags''? How might this data type help you analyze and
categorize the hashtags later? (\emph{narrative type question, no code
required})

\textbf{Solution:} I would use the `character' data type for the
variable `postHashtags'. Using this data type allows me to store many
hashtags for each post as elements within a single list. Using this data
type also allows for categorization by looking for recurring words or
phrases. By going through the list, patterns can be identified which
make it easier to group and later analyse posts.

\hypertarget{question-3-time-travelers-log}{%
\paragraph{Question 3: Time Traveler's
Log}\label{question-3-time-travelers-log}}

You're examining the timing of user interactions on a website. Would you
use a numeric or non-numeric data type to represent the timestamp of
each interaction? Explain your choice (\emph{narrative type question, no
code required})

\textbf{Solution:} I would use a `numeric' data type. Numeric timestamps
allow for a greater precision when dealing with dates and times. In the
context of the website, it will usually involve a large amount of
traffic that comes and goes rapidly, so the precision offered is
important to carry out useful and relevant analysis. Additionally, the
ease of storing all the numeric timestamp data allows for an easier
comparison of data and for identifying different types of patterns.

\hypertarget{question-4-event-elegance}{%
\paragraph{Question 4: Event Elegance}\label{question-4-event-elegance}}

You're managing an event database that includes the date and time of
each session. What data type(s) would you use to represent the session
date and time? (\emph{narrative type question, no code required})

\textbf{Solution:} I would use a `numeric' data type. Numeric timestamps
allow for a greater precision when dealing with dates and times.

\hypertarget{question-5-nominee-nominations}{%
\paragraph{Question 5: Nominee
Nominations}\label{question-5-nominee-nominations}}

You're analyzing nominations for an online award. Each participant can
nominate multiple candidates. What data type would be suitable for
storing the list of nominated candidates for each participant?
(\emph{narrative type question, no code required})

\textbf{Solution:} A `character' data type would be suitable for storing
the list of nominated candidates for each participant. The names of
nominated candidates consist of non-numeric string of characters, which
would make a numeric data type unsuitable.

\hypertarget{question-6-communication-channels}{%
\paragraph{Question 6: Communication
Channels}\label{question-6-communication-channels}}

In a survey about preferred communication channels, respondents choose
from options like ``email,'' ``phone,'' or ``social media.'' What data
type would you assign to the variable ``preferredChannel''?
(\emph{narrative type question, no code required})

\textbf{Solution:} I would assign the `character' data type to the
variable `preferredChannel'.

\hypertarget{question-7-colorful-commentary}{%
\paragraph{Question 7: Colorful
Commentary}\label{question-7-colorful-commentary}}

In a design feedback survey, participants are asked to describe their
feelings about a website using color names (e.g., ``warm red,'' ``cool
blue''). What data type would you choose for the variable
``feedbackColor''? (\emph{narrative type question, no code required})

\textbf{Solution:} I would assign the `character' data type to the
variable `feedbackColor'.

\hypertarget{question-8-variable-exploration}{%
\paragraph{Question 8: Variable
Exploration}\label{question-8-variable-exploration}}

Imagine you're conducting a study on social media usage. Identify three
variables related to this study, and specify their data types in R.
Classify each variable as either numeric or non-numeric.

\textbf{Solution:}

Variable 1: Age Data Type: Numeric (Integer)

Age is a numeric value, used to measure the age of a respondent. To
ensure consistent and easy analysis, an integer value should be used,
measuring the respondent's age in years.

Variable 2: PostsPerDay Data Type: Numeric (Double)

PostsPerDay is a numeric value, used to measure the average number of
posts made by a respondent on a daily basis. Since it is an average
value, it can and most likely will include decimal values. Thus, an
integer would not be suitable and a double should be used instead.

Variable 3: TimeSpent Data Type: Numeric (Double)

TimeSpent is a numeric value, used to measure the average time spent on
various social media platforms on a daily basis. Since it is an average
value, it can and most likely will include decimal values. Thus, an
integer would not be suitable and a double should be used instead.

\hypertarget{question-9-vector-variety}{%
\paragraph{Question 9: Vector Variety}\label{question-9-vector-variety}}

Create a numeric vector named ``ages'' containing the ages of five
people: 25, 30, 22, 28, and 33. Print the vector.

\textbf{Solution:}

\begin{Shaded}
\begin{Highlighting}[]
\CommentTok{\# Enter code here}

\NormalTok{ages }\OtherTok{\textless{}{-}} \FunctionTok{c}\NormalTok{(}\DecValTok{25}\NormalTok{,}\DecValTok{30}\NormalTok{,}\DecValTok{22}\NormalTok{,}\DecValTok{28}\NormalTok{,}\DecValTok{33}\NormalTok{)}

\FunctionTok{print}\NormalTok{(ages)}
\end{Highlighting}
\end{Shaded}

\begin{verbatim}
## [1] 25 30 22 28 33
\end{verbatim}

\hypertarget{question-10-list-logic}{%
\paragraph{Question 10: List Logic}\label{question-10-list-logic}}

Construct a list named ``student\_info'' that contains the following
elements:

\begin{itemize}
\item
  A character vector of student names: ``Alice,'' ``Bob,'' ``Catherine''
\item
  A numeric vector of their respective scores: 85, 92, 78
\item
  A logical vector indicating if they passed the exam: TRUE, TRUE, FALSE
\end{itemize}

Print the list.

\textbf{Solution:}

\begin{Shaded}
\begin{Highlighting}[]
\CommentTok{\# Enter code here}

\NormalTok{student\_info}\OtherTok{\textless{}{-}}\FunctionTok{list}\NormalTok{(}\AttributeTok{studentname=}\FunctionTok{c}\NormalTok{(}\StringTok{"Alice"}\NormalTok{,}\StringTok{"Bob"}\NormalTok{,}\StringTok{"Catherine"}\NormalTok{), }\AttributeTok{studentscore=}\FunctionTok{c}\NormalTok{(}\DecValTok{85}\NormalTok{,}\DecValTok{92}\NormalTok{,}\DecValTok{78}\NormalTok{), }\AttributeTok{studentpassed=}\FunctionTok{c}\NormalTok{(}\ConstantTok{TRUE}\NormalTok{,}\ConstantTok{TRUE}\NormalTok{,}\ConstantTok{FALSE}\NormalTok{))}

\FunctionTok{print}\NormalTok{(student\_info)}
\end{Highlighting}
\end{Shaded}

\begin{verbatim}
## $studentname
## [1] "Alice"     "Bob"       "Catherine"
## 
## $studentscore
## [1] 85 92 78
## 
## $studentpassed
## [1]  TRUE  TRUE FALSE
\end{verbatim}

\hypertarget{question-11-type-tracking}{%
\paragraph{Question 11: Type Tracking}\label{question-11-type-tracking}}

You have a vector ``data'' containing the values 10, 15.5, ``20'', and
TRUE. Determine the data types of each element using the typeof()
function.

\textbf{Solution:}

\begin{Shaded}
\begin{Highlighting}[]
\CommentTok{\# Enter code here}

\NormalTok{data }\OtherTok{\textless{}{-}} \FunctionTok{c}\NormalTok{(}\DecValTok{10}\NormalTok{, }\FloatTok{15.5}\NormalTok{, }\StringTok{"20"}\NormalTok{, }\ConstantTok{TRUE}\NormalTok{)}

\FunctionTok{typeof}\NormalTok{(data[}\DecValTok{1}\NormalTok{])}
\end{Highlighting}
\end{Shaded}

\begin{verbatim}
## [1] "character"
\end{verbatim}

\begin{Shaded}
\begin{Highlighting}[]
\FunctionTok{typeof}\NormalTok{(data[}\DecValTok{2}\NormalTok{])}
\end{Highlighting}
\end{Shaded}

\begin{verbatim}
## [1] "character"
\end{verbatim}

\begin{Shaded}
\begin{Highlighting}[]
\FunctionTok{typeof}\NormalTok{(data[}\DecValTok{3}\NormalTok{])}
\end{Highlighting}
\end{Shaded}

\begin{verbatim}
## [1] "character"
\end{verbatim}

\begin{Shaded}
\begin{Highlighting}[]
\FunctionTok{typeof}\NormalTok{(data[}\DecValTok{4}\NormalTok{])}
\end{Highlighting}
\end{Shaded}

\begin{verbatim}
## [1] "character"
\end{verbatim}

\hypertarget{question-12-coercion-chronicles}{%
\paragraph{Question 12: Coercion
Chronicles}\label{question-12-coercion-chronicles}}

You have a numeric vector ``prices'' with values 20.5, 15, and ``25''.
Use explicit coercion to convert the last element to a numeric data
type. Print the updated vector.

\textbf{Solution:}

\begin{Shaded}
\begin{Highlighting}[]
\CommentTok{\# Enter code here}

\NormalTok{prices }\OtherTok{\textless{}{-}} \FunctionTok{c}\NormalTok{(}\FloatTok{20.5}\NormalTok{,}\DecValTok{15}\NormalTok{,}\StringTok{"25"}\NormalTok{)}

\NormalTok{prices }\OtherTok{\textless{}{-}} \FunctionTok{as.numeric}\NormalTok{(prices)}

\FunctionTok{typeof}\NormalTok{(prices)}
\end{Highlighting}
\end{Shaded}

\begin{verbatim}
## [1] "double"
\end{verbatim}

\hypertarget{question-13-implicit-intuition}{%
\paragraph{Question 13: Implicit
Intuition}\label{question-13-implicit-intuition}}

Combine the numeric vector c(5, 10, 15) with the character vector
c(``apple'', ``banana'', ``cherry''). What happens to the data types of
the combined vector? Explain the concept of implicit coercion.

\textbf{Solution:}

\begin{Shaded}
\begin{Highlighting}[]
\CommentTok{\# Enter code here}

\NormalTok{x1 }\OtherTok{\textless{}{-}} \FunctionTok{c}\NormalTok{(}\DecValTok{5}\NormalTok{,}\DecValTok{10}\NormalTok{,}\DecValTok{15}\NormalTok{)}
\NormalTok{x2 }\OtherTok{\textless{}{-}} \FunctionTok{c}\NormalTok{(}\StringTok{"apple"}\NormalTok{,}\StringTok{"banana"}\NormalTok{,}\StringTok{"cherry"}\NormalTok{)}
\NormalTok{x3 }\OtherTok{\textless{}{-}} \FunctionTok{c}\NormalTok{(x1,x2)}

\FunctionTok{print}\NormalTok{(x3)}
\end{Highlighting}
\end{Shaded}

\begin{verbatim}
## [1] "5"      "10"     "15"     "apple"  "banana" "cherry"
\end{verbatim}

An implicit coercion is an automatic conversion of values by r from one
datatype to another.

\hypertarget{question-14-coercion-challenges}{%
\paragraph{Question 14: Coercion
Challenges}\label{question-14-coercion-challenges}}

You have a vector ``numbers'' with values 7, 12.5, and ``15.7''.
Calculate the sum of these numbers. Will R automatically handle the data
type conversion? If not, how would you handle it?

\textbf{Solution:}

\begin{Shaded}
\begin{Highlighting}[]
\CommentTok{\# Enter code here}

\NormalTok{numbers }\OtherTok{\textless{}{-}} \FunctionTok{c}\NormalTok{(}\DecValTok{7}\NormalTok{,}\FloatTok{12.5}\NormalTok{,}\StringTok{"15.7"}\NormalTok{)}

\NormalTok{numbers }\OtherTok{\textless{}{-}} \FunctionTok{as.numeric}\NormalTok{(numbers)}

\FunctionTok{sum}\NormalTok{(numbers)}
\end{Highlighting}
\end{Shaded}

\begin{verbatim}
## [1] 35.2
\end{verbatim}

R will not automatically handle the data type conversion. The 3rd
element in the vector `numbers' is ``15.7'', which is a character and
not a double. As a result, an error message is seen when a non-numeric
argument is used for calculating the sum of elements in a vector.

\hypertarget{question-15-coercion-consequences}{%
\paragraph{Question 15: Coercion
Consequences}\label{question-15-coercion-consequences}}

Suppose you want to calculate the average of a vector ``grades'' with
values 85, 90.5, and ``75.2''. If you directly calculate the mean using
the mean() function, what result do you expect? How might you ensure
accurate calculation?

\textbf{Solution:}

\begin{Shaded}
\begin{Highlighting}[]
\CommentTok{\# Enter code here}

\NormalTok{grades }\OtherTok{\textless{}{-}} \FunctionTok{c}\NormalTok{(}\DecValTok{85}\NormalTok{,}\FloatTok{90.5}\NormalTok{,}\StringTok{"75.2"}\NormalTok{)}

\NormalTok{grades }\OtherTok{\textless{}{-}} \FunctionTok{as.numeric}\NormalTok{(grades)}

\FunctionTok{mean}\NormalTok{(grades)}
\end{Highlighting}
\end{Shaded}

\begin{verbatim}
## [1] 83.56667
\end{verbatim}

You will receive a `NA' message. R will not be able to calculate the
mean of the vector `grades' as the 3rd element, ``75.2'' is a character,
which is non-numeric.

\hypertarget{question-16-data-diversity-in-lists}{%
\paragraph{Question 16: Data Diversity in
Lists}\label{question-16-data-diversity-in-lists}}

Create a list named ``mixed\_data'' with the following components:

\begin{itemize}
\item
  A numeric vector: 10, 20, 30
\item
  A character vector: ``red'', ``green'', ``blue''
\item
  A logical vector: TRUE, FALSE, TRUE
\end{itemize}

Calculate the mean of the numeric vector within the list.

\textbf{Solution:}

\begin{Shaded}
\begin{Highlighting}[]
\CommentTok{\# Enter code here}

\NormalTok{mixed\_data}\OtherTok{\textless{}{-}}\FunctionTok{list}\NormalTok{(}\AttributeTok{number=}\FunctionTok{c}\NormalTok{(}\DecValTok{10}\NormalTok{,}\DecValTok{20}\NormalTok{,}\DecValTok{30}\NormalTok{), }\AttributeTok{colour=}\FunctionTok{c}\NormalTok{(}\StringTok{"red"}\NormalTok{,}\StringTok{"green"}\NormalTok{,}\StringTok{"blue"}\NormalTok{), }\AttributeTok{logic=}\FunctionTok{c}\NormalTok{(}\ConstantTok{TRUE}\NormalTok{,}\ConstantTok{FALSE}\NormalTok{,}\ConstantTok{TRUE}\NormalTok{))}

\FunctionTok{mean}\NormalTok{(mixed\_data}\SpecialCharTok{$}\NormalTok{number)}
\end{Highlighting}
\end{Shaded}

\hypertarget{question-17-list-logic-follow-up}{%
\paragraph{Question 17: List Logic
Follow-up}\label{question-17-list-logic-follow-up}}

Using the ``student\_info'' list from Question 10, extract and print the
score of the student named ``Bob.''

\textbf{Solution:}

\begin{Shaded}
\begin{Highlighting}[]
\CommentTok{\# Enter code here}

\NormalTok{student\_info}\OtherTok{\textless{}{-}}\FunctionTok{list}\NormalTok{(}\AttributeTok{studentname=}\FunctionTok{c}\NormalTok{(}\StringTok{"Alice"}\NormalTok{,}\StringTok{"Bob"}\NormalTok{,}\StringTok{"Catherine"}\NormalTok{), }\AttributeTok{studentscore=}\FunctionTok{c}\NormalTok{(}\DecValTok{85}\NormalTok{,}\DecValTok{92}\NormalTok{,}\DecValTok{78}\NormalTok{), }\AttributeTok{studentpassed=}\FunctionTok{c}\NormalTok{(}\ConstantTok{TRUE}\NormalTok{,}\ConstantTok{TRUE}\NormalTok{,}\ConstantTok{FALSE}\NormalTok{))}

\FunctionTok{print}\NormalTok{(student\_info}\SpecialCharTok{$}\NormalTok{studentscore[}\DecValTok{2}\NormalTok{])}
\end{Highlighting}
\end{Shaded}

\begin{verbatim}
## [1] 92
\end{verbatim}

\hypertarget{question-18-dynamic-access}{%
\paragraph{Question 18: Dynamic
Access}\label{question-18-dynamic-access}}

Create a numeric vector values with random values. Write R code to
dynamically access and print the last element of the vector, regardless
of its length.

\textbf{Solution:}

\begin{Shaded}
\begin{Highlighting}[]
\CommentTok{\# Enter code here}

\NormalTok{nnumeric }\OtherTok{\textless{}{-}} \FunctionTok{c}\NormalTok{(}\DecValTok{1}\NormalTok{,}\DecValTok{2}\NormalTok{,}\DecValTok{3}\NormalTok{,}\DecValTok{4}\NormalTok{,}\DecValTok{5}\NormalTok{,}\DecValTok{6}\NormalTok{,}\DecValTok{7}\NormalTok{,}\DecValTok{8}\NormalTok{)}

\FunctionTok{print}\NormalTok{(nnumeric[}\FunctionTok{length}\NormalTok{(nnumeric)])}
\end{Highlighting}
\end{Shaded}

\begin{verbatim}
## [1] 8
\end{verbatim}

\hypertarget{question-19-multiple-matches}{%
\paragraph{Question 19: Multiple
Matches}\label{question-19-multiple-matches}}

You have a character vector words \textless- c(``apple'', ``banana'',
``cherry'', ``apple''). Write R code to find and print the indices of
all occurrences of the word ``apple.''

\textbf{Solution:}

\begin{Shaded}
\begin{Highlighting}[]
\CommentTok{\# Enter code here}

\NormalTok{words }\OtherTok{\textless{}{-}} \FunctionTok{c}\NormalTok{(}\StringTok{"apple"}\NormalTok{, }\StringTok{"banana"}\NormalTok{, }\StringTok{"cherry"}\NormalTok{, }\StringTok{"apple"}\NormalTok{)}

\FunctionTok{which}\NormalTok{(words }\SpecialCharTok{==} \StringTok{"apple"}\NormalTok{)}
\end{Highlighting}
\end{Shaded}

\begin{verbatim}
## [1] 1 4
\end{verbatim}

\hypertarget{question-20-conditional-capture}{%
\paragraph{Question 20: Conditional
Capture}\label{question-20-conditional-capture}}

Assume you have a vector ages containing the ages of individuals. Write
R code to extract and print the ages of individuals who are older than
30.

\textbf{Solution:}

\begin{Shaded}
\begin{Highlighting}[]
\CommentTok{\# Enter code here}

\NormalTok{ages }\OtherTok{\textless{}{-}} \FunctionTok{c}\NormalTok{(}\DecValTok{28}\NormalTok{,}\DecValTok{29}\NormalTok{,}\DecValTok{30}\NormalTok{,}\DecValTok{31}\NormalTok{,}\DecValTok{32}\NormalTok{,}\DecValTok{33}\NormalTok{,}\DecValTok{34}\NormalTok{)}

\NormalTok{ages[ages}\SpecialCharTok{\textgreater{}}\DecValTok{30}\NormalTok{]}
\end{Highlighting}
\end{Shaded}

\begin{verbatim}
## [1] 31 32 33 34
\end{verbatim}

\hypertarget{question-21-extract-every-nth}{%
\paragraph{Question 21: Extract Every
Nth}\label{question-21-extract-every-nth}}

Given a numeric vector sequence \textless- 1:20, write R code to extract
and print every third element of the vector.

\textbf{Solution:}

\begin{Shaded}
\begin{Highlighting}[]
\CommentTok{\# Enter code here}

\NormalTok{sequence }\OtherTok{\textless{}{-}}\DecValTok{1}\SpecialCharTok{:}\DecValTok{20}
\NormalTok{sequence[}\FunctionTok{seq}\NormalTok{(}\AttributeTok{from=}\DecValTok{3}\NormalTok{,}\AttributeTok{to=}\DecValTok{20}\NormalTok{,}\AttributeTok{by=}\DecValTok{3}\NormalTok{)]}
\end{Highlighting}
\end{Shaded}

\begin{verbatim}
## [1]  3  6  9 12 15 18
\end{verbatim}

\hypertarget{question-22-range-retrieval}{%
\paragraph{Question 22: Range
Retrieval}\label{question-22-range-retrieval}}

Create a numeric vector numbers with values from 1 to 10. Write R code
to extract and print the values between the fourth and eighth elements.

\textbf{Solution:}

\begin{Shaded}
\begin{Highlighting}[]
\CommentTok{\# Enter code here}

\NormalTok{sequence }\OtherTok{\textless{}{-}}\DecValTok{1}\SpecialCharTok{:}\DecValTok{10}
\FunctionTok{seq}\NormalTok{(}\AttributeTok{from=}\DecValTok{4}\NormalTok{,}\AttributeTok{to=}\DecValTok{10}\NormalTok{,}\AttributeTok{by=}\DecValTok{4}\NormalTok{)}
\end{Highlighting}
\end{Shaded}

\begin{verbatim}
## [1] 4 8
\end{verbatim}

\hypertarget{question-23-missing-matters}{%
\paragraph{Question 23: Missing
Matters}\label{question-23-missing-matters}}

Suppose you have a numeric vector data \textless- c(10, NA, 15, 20).
Write R code to check if the second element of the vector is missing
(NA).

\textbf{Solution:}

\begin{Shaded}
\begin{Highlighting}[]
\CommentTok{\# Enter code here}

\NormalTok{data }\OtherTok{\textless{}{-}} \FunctionTok{c}\NormalTok{(}\DecValTok{10}\NormalTok{,}\ConstantTok{NA}\NormalTok{,}\DecValTok{15}\NormalTok{,}\DecValTok{20}\NormalTok{)}
\FunctionTok{print}\NormalTok{(}\FunctionTok{is.na}\NormalTok{(data[}\DecValTok{2}\NormalTok{]))}
\end{Highlighting}
\end{Shaded}

\begin{verbatim}
## [1] TRUE
\end{verbatim}

\hypertarget{question-24-temperature-extremes}{%
\paragraph{Question 24: Temperature
Extremes}\label{question-24-temperature-extremes}}

Assume you have a numeric vector temperatures with daily temperatures.
Create a logical vector hot\_days that flags days with temperatures
above 90 degrees Fahrenheit. Print the total number of hot days.

\textbf{Solution:}

\begin{Shaded}
\begin{Highlighting}[]
\CommentTok{\# Enter code here}

\NormalTok{temperatures }\OtherTok{\textless{}{-}} \FunctionTok{c}\NormalTok{(}\DecValTok{90}\NormalTok{,}\DecValTok{100}\NormalTok{,}\DecValTok{60}\NormalTok{,}\DecValTok{70}\NormalTok{,}\DecValTok{120}\NormalTok{,}\DecValTok{130}\NormalTok{)}
\NormalTok{hot\_days }\OtherTok{\textless{}{-}}\NormalTok{ temperatures}\SpecialCharTok{\textgreater{}}\DecValTok{90}
\FunctionTok{sum}\NormalTok{(hot\_days)}
\end{Highlighting}
\end{Shaded}

\begin{verbatim}
## [1] 3
\end{verbatim}

\hypertarget{question-25-string-selection}{%
\paragraph{Question 25: String
Selection}\label{question-25-string-selection}}

Given a character vector fruits containing fruit names, create a logical
vector long\_names that identifies fruits with names longer than 6
characters. Print the long fruit names.

\textbf{Solution:}

\begin{Shaded}
\begin{Highlighting}[]
\CommentTok{\# Enter code here}

\NormalTok{fruits }\OtherTok{\textless{}{-}} \FunctionTok{c}\NormalTok{(}\StringTok{"banana"}\NormalTok{,}\StringTok{"strawberry"}\NormalTok{,}\StringTok{"watermelon"}\NormalTok{,}\StringTok{"pear"}\NormalTok{,}\StringTok{"pineapple"}\NormalTok{)}

\NormalTok{long\_names }\OtherTok{\textless{}{-}} \FunctionTok{nchar}\NormalTok{(fruits)}\SpecialCharTok{\textgreater{}}\DecValTok{6}

\NormalTok{fruits[long\_names]}
\end{Highlighting}
\end{Shaded}

\hypertarget{question-26-data-divisibility}{%
\paragraph{Question 26: Data
Divisibility}\label{question-26-data-divisibility}}

Given a numeric vector numbers, create a logical vector divisible\_by\_5
to indicate numbers that are divisible by 5. Print the numbers that
satisfy this condition.

\textbf{Solution:}

\begin{Shaded}
\begin{Highlighting}[]
\CommentTok{\# Enter code here}

\NormalTok{numeric }\OtherTok{\textless{}{-}} \FunctionTok{c}\NormalTok{(}\DecValTok{1}\NormalTok{,}\DecValTok{2}\NormalTok{,}\DecValTok{3}\NormalTok{,}\DecValTok{4}\NormalTok{,}\DecValTok{5}\NormalTok{,}\DecValTok{10}\NormalTok{,}\DecValTok{20}\NormalTok{,}\DecValTok{40}\NormalTok{,}\DecValTok{100}\NormalTok{,}\DecValTok{99}\NormalTok{)}
\NormalTok{divisible\_by\_5 }\OtherTok{\textless{}{-}}\NormalTok{ (numeric }\SpecialCharTok{\%\%} \DecValTok{5}\NormalTok{) }\SpecialCharTok{==} \DecValTok{0} 
\NormalTok{numeric[divisible\_by\_5]}
\end{Highlighting}
\end{Shaded}

\begin{verbatim}
## [1]   5  10  20  40 100
\end{verbatim}

\hypertarget{question-27-bigger-or-smaller}{%
\paragraph{Question 27: Bigger or
Smaller?}\label{question-27-bigger-or-smaller}}

You have two numeric vectors vector1 and vector2. Create a logical
vector comparison to indicate whether each element in vector1 is greater
than the corresponding element in vector2. Print the comparison results.

\textbf{Solution:}

\begin{Shaded}
\begin{Highlighting}[]
\CommentTok{\# Enter code here}

\NormalTok{vector1 }\OtherTok{\textless{}{-}} \FunctionTok{c}\NormalTok{(}\DecValTok{2}\NormalTok{,}\DecValTok{4}\NormalTok{,}\DecValTok{6}\NormalTok{,}\DecValTok{51}\NormalTok{,}\DecValTok{69}\NormalTok{,}\DecValTok{99}\NormalTok{)}
\NormalTok{vector2 }\OtherTok{\textless{}{-}} \FunctionTok{c}\NormalTok{(}\DecValTok{3}\NormalTok{,}\DecValTok{12}\NormalTok{,}\DecValTok{15}\NormalTok{,}\DecValTok{27}\NormalTok{,}\DecValTok{31}\NormalTok{,}\DecValTok{84}\NormalTok{)}
\NormalTok{greater\_vec }\OtherTok{\textless{}{-}}\NormalTok{ vector1 }\SpecialCharTok{\textgreater{}}\NormalTok{ vector2 }
\FunctionTok{print}\NormalTok{(greater\_vec)}
\end{Highlighting}
\end{Shaded}

\begin{verbatim}
## [1] FALSE FALSE FALSE  TRUE  TRUE  TRUE
\end{verbatim}

\end{document}
